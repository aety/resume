%%%%%%%%%%%%%%%%%%%%%%%%%%%%%%%%%%%%%%%%%
% Important note:
% This template requires the resume.cls file to be in the same directory as the
% .tex file. The resume.cls file provides the resume style used for structuring the
% document.
%
%%%%%%%%%%%%%%%%%%%%%%%%%%%%%%%%%%%%%%%%%

%----------------------------------------------------------------------------------------
%	PACKAGES AND OTHER DOCUMENT CONFIGURATIONS
%----------------------------------------------------------------------------------------

\documentclass{resume} % Use the custom resume.cls style

\usepackage[left=0.5in,top=0.5in,right=0.5in,bottom=0.5in]{geometry} 
\usepackage{hyperref}
\usepackage{enumitem}
\renewcommand{\familydefault}{\sfdefault}
\usepackage{helvet}

% Document margins
\newcommand{\tab}[1]{\hspace{.2667\textwidth}\rlap{#1}}
\newcommand{\itab}[1]{\hspace{0em}\rlap{#1}}
%\usepackage{helvet}
%\renewcommand{\familydefault}{\sfdefault}
\name{ANNE EN-TZU YANG} % Your name
\address{Minneapolis, MN $\vert$ anneyanget@gmail.com $\vert$ (617) 309-9419} % Your address
\address{\href{http://github.com/aety}{github.com/aety} $\vert$ \href{http://linkedin.com/in/aetyang}{linkedin.com/in/aetyang} $\vert$ \href{http://sites.google.com/view/aety}{sites.google.com/view/aety}} % Your phone number and email

\begin{document}
	
	%----------------------------------------------------------------------------------------
	%	EDUCATION SECTION
	%----------------------------------------------------------------------------------------
	
	
	
	%----------------------------------------------------------------------------------------
	%	EXPERIENCE SECTION
	%----------------------------------------------------------------------------------------
	\begin{rSection}{Experience}
		
		\begin{itemize}[leftmargin=0em]
			\item {\bf Data Science Fellow}{. Insight Data Science \textit{(Minneapolis, MN)}} \hfill {\em 09/2019 - present}\\
			\vspace{-5mm}
			\begin{itemize}
				\setlength\itemsep{-1.75em}
				\item Deployed an \textit{html} web app recommending best time to ride Paris metro based on air quality prediction.\\
				\item Utilized \textit{Prophet} to predict hourly PM10 (pollutant) concentration, with an SMAPE error of 12\% \\
				\item Forecast results are stored on \textit{AWS} in \textit{PostgreSQL} for web app queries via \textit{Flask}.\\
				\item Identified predictors correlated to air quality by R\textsuperscript{2} = 0.96 using \textit{scikit-learn's random forest regressor}.\\
				\item Visualized results as \textit{Google Charts} \href{https://tinyurl.com/yxptcaz7
				}{figures} to provide intuitive information for health risks management.
			\end{itemize}	
			
			\item {\bf Postdoctoral Researcher}{. Inst. for Intelligent Systems and Robotics \textit{(Paris, France)}} \hfill {\em 09/2018 - 08/2019}\\
			\vspace{-5mm}
			\begin{itemize}
				\setlength\itemsep{-1.75em}
				\item  Designed a system of \href{https://tinyurl.com/yxk4jou4}{helical markers} that enabled the 3D tracking of intraoperative surgical tools from individual 2D X-ray images.\\ 
				\item  Trained \textit{convolutional neural networks} to successfully reconstruct deformable 3D shape and orientation at $\sim$ 10 ms/frame (errors $<$1$^\circ$) with medical (\textit{DICOM}) images acquired from an operating room.
			\end{itemize}
			
			\item {\bf PhD Intern}{. Sanofi \textit{(Bridgewater, NJ)}} \hfill {\em 06/2017 - 08/2017}\\
			\vspace{-5mm}
			\begin{itemize}
				\setlength\itemsep{-1.75em}	
				\item Collaborated with pharmacologists and immunologists on adding a new module to existing computational model to simulate periostin (protein) in asthma formation and treatment.\\
				\item Wrote \textit{Matlab} scripts to automate statistical tests and data visualization to expedite data analysis on 10k entries of clinical trial data.
			\end{itemize}
			
			\item {\bf PhD Candidate}{. Northwestern University \textit{(Evanston, IL)}} \hfill {\em 09/2012 - 08/2018}\\
			\vspace{-5mm}
			\begin{itemize}
				\setlength\itemsep{-1.75em}
				\item Investigated interdisciplinarily the neural pathway of rat whiskers to understand human's sense of touch.\\
				\item Created a \href{https://www.youtube.com/watch?v=EPuThXPd-qw}{\textit{MEMS}-sensor} able to detect mechanical signals on a rat whisker of $<$200 $\mu$m diameter.\\
				\item Initiated a multi-university collaboration that later won a \$1M NSF \href{https://www.nsf.gov/news/mmg/mmg_disp.jsp?med_id=132588}{grant}.\\
				\item Constructed static and dynamic models of tapered beams in \textit{Matlab} and \textit{Python} to quantify forces and moments on the whiskers when undergoing contact or airflow.\\
				\item Predicted the timing and magnitude of 4 categories of neural responses (R\textsuperscript{2}=0.93) from 420 sets of 100-ms data sampled at 10kHz.\\
				\item Analyzed data of $>$500 rat whiskers and built predictive models of whisker geometry by whisker identity.	
			\end{itemize}
			
			
		\end{itemize}
		
	\end{rSection}
	
	
	
	%----------------------------------------------------------------------------------------
	%	EDUCATION SECTION
	%----------------------------------------------------------------------------------------
	\begin{rSection}{Education}
		%--copy and paste this region  if you need more--
		\begin{itemize}[leftmargin=0em]
			\item {\bf PhD}{. Northwestern University} \textit{(Evanston, IL)} \hfill {\em 09/2012 - 08/2018} 
			\vspace{-3mm}
			\begin{itemize}
				\setlength\itemsep{-3em}
				\item Mechanical Engineering
			\end{itemize}
			\item {\bf Certificate}{. Kellogg School of Management} \textit{(Evanston, IL)} \hfill {\em 06/2016 - 08/2016} 
			\vspace{-3mm}
			\begin{itemize}
				\setlength\itemsep{-3em}
				\item Management for Scientists and Engineers
			\end{itemize}
			\item {\bf BS}{. National Taiwan University} \textit{(Taipei, Taiwan)} \hfill {\em 09/2008 - 06/2012} 
			\vspace{-3mm}
			\begin{itemize}
				\setlength\itemsep{-3em}
				\item Mechanical Engineering
			\end{itemize}
		\end{itemize}
		
		
		
		
		%--copy and paste this region  if you need more--
		
	\end{rSection}
	
	%----------------------------------------------------------------------------------------
	%	SKILLS SECTION
	%----------------------------------------------------------------------------------------
	\begin{rSection}{Skills}
		\begin{itemize}[leftmargin=0em]
			\item {\bf Languages}{. Matlab, Python, SQL, LaTeX, HTML, Javascript}
			\item {\bf Packages}{.Pandas, Flask, Numpy, Scipy, scikit-learn, statsmodels, BeautifulSoup, Prophet, PostgreSQL, \\SQLAlchemy, matplotlib, Google Developers Charts, Matlab regionprops, Matlab nftool}
			\item {\bf Tools}{. Git, Github, Jupyter Notebook, AWS (RDS, EC2, Route 53), Linux, API, 3Dslicer}
			%\item {\bf Knowledge}{. medical imaging (DICOM), machine learning (convolutional neural network, neural netwrok regression, random forest regression), autoregressive integrated moving average}
			
		\end{itemize}
	\end{rSection}
	
	
	
	
\end{document}----------------------------

